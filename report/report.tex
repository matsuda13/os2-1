\documentclass[11pt, a4paper]{jsarticle}
\usepackage[dvipdfmx]{graphicx}
\begin{document}
\noindent
operating system 問題2.1\\
報告者:205759A 比嘉和樹\\
提出日:2021/1/18

\section{実験前の考察}
ファイルシステムは OS とは独立しているが、現在多くの OS がサポートしている。これはメモ
リを分割してファイルとして割り当て、ディレクトリの構造の概念をメモリに持たせるものであ
る。go のプログラムは package os を使用することで OS と連携し、ファイルへの書き込みを実現
する。package os の File 構造体には Write 関数が設定されており、これは byte 型の引数に含ま
れる文字列を File 構造体で指定したファイルに書き込む。このとき引数とする byte 型で指定した
データを一気に OS に渡し、ファイル I/O 処理の回数を減らすのが golang のファイル書き込みに
おけるバッファリングの手法である。このことから、バッファのサイズを大きくするごとに、ファ
イル I/O の回数はより減ることになり、処理時間はその分だけ短くなることと、ファイルサイズ
が大きくなるごとにその影響は顕著となることが予想できる。
今回の実験では、$2^2$~$2^{12}$ のバッファサイズでデータファイルの大きさが 5MB になるように書
き込みを行い、その処理時間を測る手順を 10 回繰り返した。その後、各バッファサイズについて
処理時間の平均値を求め、1MB を書き込む秒数として処理速度を求めた。図 1 は、そのグラフで
ある。
\begin{figure}[htbp]
	\centering
	\includegraphics[width=120mm]{../out/out.jpg}
\end{figure}
\end{document}